\section{Data Preparation}

Our data mining process intends to use the program Vowpal Wabbit (which will be explained further in the section to follow). Vowpal Wabbit,
or VW, takes a custom inout format which needed to be accounted for before any data mining could take place. While preparing this custom
data format, various other steps were also taken (to be explained) in order to prepare the data for various attemps and algorithms with
data mining. We will outline below the type of input format needed, the custom data transformations, as well as data normalization for
future data mining efforts.\\

The input format for VW is as follows:

\begin{equation}
\label{eq:input}
[Label] [Importance [Tag]]|Namespace Features |Namespace Features ... |Namespace Features
\end{equation}

where Namespace=String[:Value] and Features=(String[:Value] )*. Here we can see the imediate differences to the original CSV files that
are provided for the Kaggle challenge. The Label is the floating point number that is being attemped to be predicted. The importance
tage can be used to give weight to a specific training instance over the ther instances. This can be useful for weighted preditions
or higher confidence in a particular instance. Namespace is a value identifier while features are the values associated with the identifier.
In order to transform the CSV files to the given VW input format, python scripts were created and run to make the conversion. While
these python scipts were used primarily for conversion purposes, they also allowed the tweaking and adjustments of data as it was converted.
We will now explain the different steps taken to tweak data during the conversion.\\

As it was shown on the Kaggle website, the data given in the training instanced represented a skewed bell curve. In order to get
this data into a more acceptable bell curve shape, or normalized (as was needed to VW to run properly, more on this later), two
solutions were used, one at a time. The first was simply to take the logarithm of the data set. This creates a more acceptable
normalization of the data. That being said, this was the first tweak to be given to the data during the conversion process. The other
availible option for creating normalized data is to take the sqaure root of the instances. This also gave a somewhat more normalized
view of the data. This tweak was also taken on the data although was preformed at a different time than the logarithm. The results
of these two data tweaks can be seen in Table 1 as the different types of normalization transformations.\\

The second large tweak to be placed on the data during the conversion process was the limiting of free text fields to only 
keywords. In order to make this tweak, the python library topia.termextract was used to extract keywords from local text fields.
This means that every text field was analysed sperately for keywords as opposed to looking at all training instances at once.
The keyword threshold for reoccurance was set only to 1 as we decided that job descriptions did not always repeat the most important
words at any given rate. This data tweak was only applied to data for particular runs of the data mining process and can be seen
in the results in Table 1 for how it was used in relation to other data tweaks and their outcomes.\\

The final data tweak to be placed on the data during the conversion process was the limiting of attributes to be used during 
the training instance model creation. Our python scripts allowed us to adjust which attributes were to be used for the model
generation. Here we played with one major setting in which we turned off all free text fields. The results of turning off
the free text fields can again be seen in Table 1 in relation with the normalization tweaks. A large number of combinations
of free text and categorical attributes could be tested at any given time with or without keywords with this program. However,
due to the limited time of this project only the few were actually tested.\\

This conversion of data to be handles by a larger data set data mining program along with the keyword analysis and other
data tweaks represent a non trival step in our data mining project.